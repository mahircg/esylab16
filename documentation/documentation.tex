%%%%%%%%%%%%%%%%%%%%%%%%%%%%%%%%%%%%%%%%%%%%%%%%%%%%%%%%%%%%%%%%%%%%%%%%%%%%%%%%%%%%%%
% To make formatting easy tell LaTeX what kind of document you want to write
% by changing the according {} to {#1}
%%%%%%%%%%%%%%%%%%%%%%%%%%%%%%%%%%%%%%%%%%%%%%%%%%%%%%%%%%%%%%%%%%%%%%%%%%%%%%%%%%%%%%


%%%%%%%%%%%%%%%%%%%%%%%%%%%%%%%%%%%%%%%%%%%%%%%%%%%%%%%%%%%%%%%%%%%%%%%%%%%%%%%%%%%%%%
% Which language do you want to write in
%%%%%%%%%%%%%%%%%%%%%%%%%%%%%%%%%%%%%%%%%%%%%%%%%%%%%%%%%%%%%%%%%%%%%%%%%%%%%%%%%%%%%%
%English
\newcommand{\EN}[1]{#1}
%German
\newcommand{\DE}[1]{}

%%%%%%%%%%%%%%%%%%%%%%%%%%%%%%%%%%%%%%%%%%%%%%%%%%%%%%%%%%%%%%%%%%%%%%%%%%%%%%%%%%%%%%
% For print single or double page?
%%%%%%%%%%%%%%%%%%%%%%%%%%%%%%%%%%%%%%%%%%%%%%%%%%%%%%%%%%%%%%%%%%%%%%%%%%%%%%%%%%%%%%
\newcommand{\single}[1]{#1}
\newcommand{\double}[1]{}

%%%%%%%%%%%%%%%%%%%%%%%%%%%%%%%%%%%%%%%%%%%%%%%%%%%%%%%%%%%%%%%%%%%%%%%%%%%%%%%%%%%%%%
% Now this is followed by a lot of page and command definitions. For the beginning you 
% should be able to continue where you find the next comment section like this one.
% However, you might want to take a look a the definitions sometime to be able to use 
% them. Of course you can also add the defintions you needed yourself.
%%%%%%%%%%%%%%%%%%%%%%%%%%%%%%%%%%%%%%%%%%%%%%%%%%%%%%%%%%%%%%%%%%%%%%%%%%%%%%%%%%%%%%

\single{\documentclass[12pt,a4paper,oneside,german,english]{book}}
\double{\documentclass[12pt,a4paper,twoside,german,english]{book}}

\usepackage[ngerman,english]{babel}
%\usepackage[draft,breaklinks=true,colorlinks=false,dvips,bookmarks,pdffitwindow,pdfcenterwindow=true,pdfstartview=Fit]{hyperref}
%\usepackage[breaklinks=true,colorlinks=false,dvips,bookmarks,pdffitwindow,pdfcenterwindow=true,pdfstartview=Fit]{hyperref}
\usepackage{setspace}
\usepackage{cite}
\usepackage{float,times} 
\usepackage[utf8x]{inputenc}
\usepackage[T1]{fontenc}
\usepackage{amsmath,amsthm,latexsym}
\usepackage[dvips]{epsfig}
\usepackage{subfigure}
\usepackage{rotating}

\usepackage{minted}
\usepackage{newfloat}
\usepackage{xspace}

\usepackage{xcolor}
\usepackage{afterpage}
\usepackage{multirow}
\usepackage{array}

%! put the new-command after the inputenc package use, so 'ü' will not be shown as abomination ASCII -TF
	
%%%%%%%%%%%%
%%% MINTED
%%%%%%%%%%%%

\DeclareFloatingEnvironment[
  fileext=loc,
  listname=List of Listings,
  name=Listing,
  placement=htp,
]{codefloat}

\newenvironment{vhdl}[2][]
 {\codefloat[htb]
  \if\relax\detokenize{#1}\relax\else\caption{#1}\fi
  \if\relax\detokenize{#2}\relax\else\label{#2}\fi 
  \minted[%linenos,
          %numbersep=5pt,
          tabsize=3,
          frame=lines,
          framesep=2mm,
          fontsize=\footnotesize]
         {vhdl}}
 {\endminted
 \endcodefloat}
\newenvironment{asm}[2][]
 {\codefloat[htb]
  \if\relax\detokenize{#1}\relax\else\caption{#1}\fi
  \if\relax\detokenize{#2}\relax\else\label{#2}\fi 
  \minted[%linenos,
          %numbersep=5pt,
          tabsize=3,
          frame=lines,
          framesep=2mm,
          fontsize=\footnotesize]
         {c}}
 {\endminted
 \endcodefloat}

%%%%%%%%%%%%
%%% COMMENTS AND TODO-MARKERS
%%%%%%%%%%%%

% \newcommand{\comment}[3]{\marginpar{\textcolor{#2}{Comment: #1}}\textcolor{#2}{\textit{[#1: #3]}}}
% \newcommand{\TF}[1]{\comment{TF}{orange}{#1}}
% \newcommand{\LS}[1]{\comment{LS}{blue}{#1}}
% 
% \newcommand{\qn}{\todo{qn!}}
% \newcommand{\sectodo}{\todo{Still todo.}}
% \newcommand{\seccheck}{\marginpar{\textcolor{orange}{reread}}}
% \newcommand{\secthomas}{\marginpar{\textcolor{green}{thomas}}}
% \newcommand{\secdone}{\marginpar{\textcolor{gray}{done}}}
% %\newcommand{\todo}[1]{\pagecolor{red!10}\afterpage{\nopagecolor}\textcolor{red}{\textit{[#1]}}\marginpar{\textcolor{red}{\textbf{TODO}}}}
% \newcommand{\todo}[1]{\textcolor{red}{\textit{[#1]}}\marginpar{\textcolor{red}{\textbf{TODO}}}}
% 
% 
% %%% BLANK COMMANDS FOR PUBLISHING
% \newcommand{\publish}[1]{}
% \renewcommand{\publish}[1]{#1} % TODO: uncomment this line before printing!
% 
% \publish{\renewcommand{\comment}[3]{}}
% \publish{\renewcommand{\TF}[1]{}}
% \publish{\renewcommand{\LS}[1]{}}
% \publish{\renewcommand{\qn}{}}
% \publish{\renewcommand{\sectodo}{}}
% \publish{\renewcommand{\seccheck}{}}
% \publish{\renewcommand{\secthomas}{}}
% \publish{\renewcommand{\secdone}{}}
% \publish{\renewcommand{\todo}[1]{}}

\newcommand{\comment}[3]{\marginpar{\textcolor{#2}{Comment: #1}}\textcolor{#2}{\textit{[#1: #3]}}}
\newcommand{\TF}[1]{\comment{TF}{red}{#1}}
\newcommand{\TS}[1]{\comment{TS}{blue}{#1}}
\newcommand{\FP}[1]{\comment{TS}{orange}{#1}}

%%%%%%%%%%%%
%%% LOGIC OPERATORS
%%%%%%%%%%%%

\newcommand{\logicxor}{\texttt{XOR}\xspace}
\newcommand{\logicor}{$||$\xspace}%TODO: does not show up as OR
\newcommand{\logicand}{\&\xspace}

%%%%%%%%%%%%
%%% STUFF
%%%%%%%%%%%%

\newcommand{\mem}[1]{*#1\xspace}
\newcommand{\x}{--\xspace}

%%%%%%%%%%%%
%%% NAMES
%%%%%%%%%%%%

\newcommand{\procname}{\textsl{LT16x32}\xspace}

%%%%%%%%%%%%
%%% FORMAT SHORTCUTS
%%%%%%%%%%%%

\newcommand{\floatplace}{[ht]}

\newcommand{\tc}[1]{\textbf{#1}}

\newcommand{\hex}[1]{#1}
\newcommand{\bits}[1]{#1}

\newcommand{\filename}[1]{{\small\texttt{#1}}}
\newcommand{\command}[1]{{\small\texttt{#1}}}
\newcommand{\inlinevhdl}[1]{{\small\texttt{#1}}} %TODO with syntax highlighting and check for $\_$
\newcommand{\inlinec}[1]{{\small\texttt{#1}}} %TODO with syntax highlighting and check for $\_$
\newcommand{\inlineasm}[1]{{\small\texttt{#1}}} %TODO with syntax highlighting and check for $\_$

\newcommand{\instruction}[7]{
\subsubsection{#1}
\label{instr_#2}
\begin{description}
\nolistskip
\item[Opcode:] \texttt{#3}
\item[Assembler:] \inlineasm{#4}
\item[Operation:] #5
\item[C-Equivalent:] \inlinec{#6}
\item[Status Register:] #7
\end{description}
}


%%%%%%%%%%%%
%%% TEX WORKAROUNDS
%%%%%%%%%%%%
\newcommand{\nolistskip}{\itemsep-3pt}

\newcommand{\DocuTitle}{Documentation to the LT16x32} %TODO: can not use \procname here, why so ever...
%\newcommand{\DiplTitleGerman}{Dokumentation für den LT16x32}

% format page layout.

\setlength{\topmargin}{0cm}
\setlength{\textwidth}{15cm}
\setlength{\textheight}{22cm}
\setlength{\oddsidemargin}{1cm}
\setlength{\evensidemargin}{0cm}

\setlength{\headheight}{15pt}
% \setlength{\voffset}{-0cm}
% \setlength{\topmargin}{0cm}
% \setlength{\headheight}{0.54cm}
% \setlength{\textheight}{23cm}
% % \setlength{\headsep}{1.5cm}
% %\setlength{\hoffset}{-2.54cm}
% \setlength{\hoffset}{0cm}
% \setlength{\oddsidemargin}{0.46cm}
% \setlength{\evensidemargin}{0.46cm}
% \setlength{\textwidth}{15cm}
% \setlength{\marginparsep}{0cm}
% \setlength{\marginparwidth}{1.54cm}

\setcounter{secnumdepth}{3}


% \setlength{\footskip}{1cm}
% \setlength{\parindent}{0cm}
% \setlength{\parskip}{1em}

\usepackage{fancyhdr}
\pagestyle{fancy}
\fancyhead{} % clear all header fields
%\fancyhead[LO, RE]{\slshape \nouppercase{\leftmark}} % chapter titles
%\fancyhead[LE, RO]{\slshape \nouppercase{\leftmark}\\\nouppercase{\rightmark}} % section titles
\double{\fancyhead[LE]{\slshape \nouppercase{\leftmark}}} % chapter titles
\fancyhead[RO]{\slshape \nouppercase{\rightmark}} % section titles
\fancyfoot{} % clear all footer fields
\fancyfoot[C]{\thepage}


\usepackage[%dvips,
	colorlinks=false,
	bookmarks,
	pdffitwindow,
	pdfcenterwindow=true,
	pdfstartview=Fitpdftex,
	pdfauthor={Lasse Schnepel},
	pdftitle={\DocuTitle},
	pdfsubject={LT16x32 Documentation}, % one sentence summery
	pdfkeywords={Processor,Firmware-based Verification,Architecture design}, %comma-seperated keywords
	pdfproducer={Latex with hyperref},
	pdfcreator={}]{hyperref}

\begin{document}

%\lhead[\fancyplain{}{\thepage}]         {\fancyplain{}{\rightmark}}
%\chead[\fancyplain{}{}]                 {\fancyplain{}{}}
%\rhead[\fancy{}{\rightmark}]       {\fancy{}{\thepage}}
%\rhead[\fancyplain{}{\rightmark}]       {\fancyplain{}{\thepage}}
%\lfoot[\fancyplain{}{}]                 {\fancyplain{\tstamp}{\tstamp}}
%\cfoot[\fancy{\thepage}{}]         {\fancy{\thepage}{}}
%\cfoot[\fancyplain{\thepage}{}]         {\fancyplain{\thepage}{}}
%\rfoot[\fancyplain{\tstamp} {\tstamp}]  {\fancyplain{}{}}
% \fancyhf{} %delete the current section for header and footer
% \fancyhead[LE,RO]{\bfseries\thepage}
% \fancyhead[LO]{\bfseries\rightmark}
% \fancyhead[RE]{\bfseries\leftmark}
% \renewcommand{\headrulewidth}{0.5pt}
% % make space for the rule
% \fancypagestyle{plain}{%
% \fancyhead{} %get rid of the headers on plain pages
% %\renewcommand{\headrulewidth}{0} % and the line
% } 

\EN{\selectlanguage{english}}
\DE{\selectlanguage{german}}


%%%%%%%%%%%%%%%%%%%%%%%%%%%%%%%%%%%%%%%%%%%%%%%%%%%%%%%%%%%%%%%%%%%%%%
% Here you have to start editing
%%%%%%%%%%%%%%%%%%%%%%%%%%%%%%%%%%%%%%%%%%%%%%%%%%%%%%%%%%%%%%%%%%%%%%
\title{\DiplTitle}
  \author{Lasse Schnepel}
\pagenumbering{arabic} 
\hyphenation{Bit-ebenen Gateprop Bit-ebene Fanin Boole-sche
  Partial-produkt-generator
  Code-inspektion
  Verifika-tions-ablauf
  IPC-Verifika-tions-ablauf
  Abhangig-keiten
}


%%%%%%%%%%%%%%%%%%%%%%%%%%%%%%%%%%%%%%%%%%%%%%%%%%%%%%%%%%%%%%%%%%%%%%%%
% title page
%%%%%%%%%%%%%%%%%%%%%%%%%%%%%%%%%%%%%%%%%%%%%%%%%%%%%%%%%%%%%%%%%%%%%%%%
\frontmatter
%%%%%%%%%%%%%%%%%%%%%%%%%%%%%%%%%%%%%%%%%%%%%%%%%%%%%%%%%%%%%%%%%%%%%%%%
% titel page
%%%%%%%%%%%%%%%%%%%%%%%%%%%%%%%%%%%%%%%%%%%%%%%%%%%%%%%%%%%%%%%%%%%%%%%%

\thispagestyle{empty}

\begin{center}
  \mbox{}
  \vspace{2cm} 
  \Large{Documentation of the \procname}
  \vspace{1cm}

\normalsize  
 
%%%%%%%%%%%%%%%%%%%%%%%%%%%%%%%%%%%%%%%%%%%%%%%%%%%%%%%%%%%%%%%%%%%%%%%%
% Put the current year here
%%%%%%%%%%%%%%%%%%%%%%%%%%%%%%%%%%%%%%%%%%%%%%%%%%%%%%%%%%%%%%%%%%%%%%%%
  Kaiserslautern, 2014 \\
  \vfill
\end{center}

\tableofcontents

\mainmatter

%%%%%%%%%%%%%%%%%%%%%%%%%%%%%%%%%%%%%%%%%%%%%%%%%%%%%%%%%%%%%%%%%%%%%%%
% Now this is followed by your chapters
% this usually starts with introduction and Fundamentals and then 
% continues with whatever you need or did
%%%%%%%%%%%%%%%%%%%%%%%%%%%%%%%%%%%%%%%%%%%%%%%%%%%%%%%%%%%%%%%%%%%%%%%

\input{content}

\backmatter
\listoffigures 
\listoftables
\listofcodefloat %triggers Errors for me, for whatever reason, but the pdf is generated nonetheless -TF

%%%%%%%%%%%%%%%%%%%%%%%%%%%%%%%%%%%%%%%%%%%%%%%%%%%%%%%%%%%%%%%%%%%%%%%%
% Choose your Bibtex Style File (here alphadin.bst) and references.
% Hint: For references make a local link refs2 to our jabref 
% directory "/import/jabref/refs2.bib"
%%%%%%%%%%%%%%%%%%%%%%%%%%%%%%%%%%%%%%%%%%%%%%%%%%%%%%%%%%%%%%%%%%%%%%%%

%%%%%%%%%%%%%%%%%%%%%%%%%%%%%%%%%%%%%%%%%%%%%%%%%%%%%%%%%%%%%%%%%%%%%%%%
\bibliographystyle{alphadin}
\bibliography{refs3}
%%%%%%%%%%%%%%%%%%%%%%%%%%%%%%%%%%%%%%%%%%%%%%%%%%%%%%%%%%%%%%%%%%%%%%%%

%%%%%%%%%%%%%%%%%%%%%%%%%%%%%%%%%%%%%%%%%%%%%%%%%%%%%%%%%%%%%%%%%%%%%%%%
% Curriculum Vitae
%%%%%%%%%%%%%%%%%%%%%%%%%%%%%%%%%%%%%%%%%%%%%%%%%%%%%%%%%%%%%%%%%%%%%%%%
%\selectlanguage{german}
%\input{cv.tex}
%\selectlanguage{english}

\end{document}
